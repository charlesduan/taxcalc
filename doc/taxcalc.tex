\documentclass[12pt]{article}

\title{Framework for Tax Preparation}
\author{Charles Duan}

\begin{document}

\maketitle

\tableofcontents

This is a software system for automating the computation and filling of tax
forms. Much like commercial software for tax preparation, the object of the
system is to allow a user to enter data from received tax forms and other
sources, and then answer interview questions, to produce completed tax forms
ready for filing.

Unlike commercial tax software, however, this system is not intended to
comprehensively cover all tax forms and options. Instead, a second object of
this system is to provide a framework to make it reasonably easy for
programming-adept users to add tax forms as necessary to their personal
situations and needs.

\section{Overview of Operation}

Use of the system proceeds in three phases.

The first phase is the computation of form values. The user enters relevant data
from received tax forms and other sources into files, and then writes a short
script that imports that data and computes the values for tax forms to be
prepared. This produces a plain text file containing the line item data to be
filled into the tax forms.

The second phase is the identification of form fields. For each tax form to be
filled in, the user provides the PDF file for the blank form and also the file
of line item data. A program displays a graphical interface where the user can
click on the blank spaces where each line item should be entered. The result of
this is a second plain text data file that lists the coordinates of each of the
form fields.

The third phase is the actual filling of the forms. The user writes another
short script that receives the data files from the first and second phases, and
places the line items from the first-phase file into the PDF form. If any
changes need to be made subsequently, the user may simply rerun the first-phase
computation script and then rerun the third-phase filling script, thereby
updating all of the tax forms automatically.




\section{Recommended Procedure for Annual Filing}

This section provides recommendations on preparing an annual tax filing. This
includes updating the software in view of changes to the tax laws and forms, as
well as preparing the returns themselves.

First, prepare the script for running the computations, and enter the data for
received forms (W-2s, 1099s, etc.). It is preferable to do this first rather
than updating the tax form computations, because it allows for partially running
the program in advance. This has two benefits as you update the computations: It
reveals bugs earlier, and it lets you assess with real data whether it is worth
implementing a new computation.

At this step, be sure to review the informational forms in the Appendix and
ensure that you have entered all relevant information required there.

Next, update the computations. Start with the main form being computed (e.g.,
Form 1040), and work line by line through the script. I found it preferable to
address each form dependency as it came immediately, resulting in a depth-first
update. That is, when the 1040 calls for a result from Schedule B, I would pause
work on the 1040 and go update Schedule B. The advantage of this is that you can
continue running partial tests of the computations. The main disadvantage is
that it becomes hard to keep track of which form you are working on; I generally
addressed this by leaving comments in the files as I worked on them.

Once all the computations are done and run without errors, move on to phase 2
and mark the form fields. I found it useful to mark both the forms and the
worksheets, even though the worksheets do not need to be filed, because it made
it easier to review everything for accuracy.

Then prepare the script for filling in all the forms (phase 3), and fill them in
to produce a draft return. This will help to identify any computation errors.

Once you have reviewed the draft return and corrected any issues, you can
optimize the return (e.g., try different filing statuses, move dependents
around, choose standard versus itemized deductions, and so on). Once you are
satisfied, you can print, sign, and file your returns.

\section{Object Classes and Data Structures}

The tax computation system is written in Ruby and is organized around two main
classes: the TaxForm and the FormManager.


\subsection{The TaxForm API}

Generally, each tax form is represented as an instance of a Ruby class for that
form, a subclass of TaxForm. A TaxForm has four essential methods:
\begin{description}
\item[name] The name of this form. The word ``Form'' should not precede names.
\item[year] The tax year for which this form has been updated. This is used to
check whether the form has been updated yet. (An out-of-date form will still
be executed; it will just produce a warning.)
\item[compute] Compute the form. This method should fill in all the lines of the
form.
\item[needed?] Whether the tax form is ultimately needed for inclusion in the
return. This method should be called after compute, and it should inspect the
results of the lines of the form to determine if the form needs to be filed.
\end{description}

A form is essentially a hash table that maps line numbers to data values. Line
numbers may be any strings so long as they contain no spaces. Values may be
atomic values (numbers, strings, dates), or may be arrays of those atomic
values. The order of lines is remembered (and can be altered manually).


\subsubsection{Referring to Lines in a Form}

The TaxForm has several methods with the name ``line'' for getting and setting
lines:
\begin{itemize}
\item line[\#] returns the value of that line. If the value is an array or is
unset, an error is raised.
\item line[\#, :opt] returns the value of the line, or if the line is
unset it returns BlankZero.
\item line[\#, :all] returns an array for this line's values. It always returns
an array, even if the line only has an atomic value.
\item line[\#, :sum] returns the sum of this line's values.
\item line[\#, :present] returns true or false depending on whether the line is
set.
\item line[\#] = x will set the line to an atomic value; it will raise an error
if given an array. If the line is already set, a warning will be raised.
\item line[\#, :all] = [ x, y, z ] will set the line to an array value.
\item line[\#, :add] = x will force the line to be an array and then will append
the given value to that array.
\item line[\#, :overwrite] = x will set the line to an atomic value, and will
not warn if the line was already set.
\end{itemize}
As a shortcut, the ``method\_missing'' method for TaxForm will infer line
numbers from methods named ``line\_\#''.

Every TaxForm is associated with a FormManager. Many of the methods of TaxForm
are delegated to its FormManager as a matter of convenience.

\subsubsection{How Tables Are Stored in Forms}

Consider a form that requires input like so:
\begin{center}
\begin{tabular}{|l|l|l|}
\hline
\textbf{Line 1a} & \textbf{1b} & \textbf{1c} \\
\textbf{Business name} & \textbf{Income} & \textbf{Expenses} \\
\hline
Acme Enterprises & 5000 & 400 \\
\hline
Foobar Inc. & 10000 & \\
\hline
Big Corp. & 15000 & 600 \\
\hline
\end{tabular}
\end{center}
This would be stored in a TaxForm as three lines named 1a, 1b, and 1c as
follows:
\begin{itemize}
\item Line 1a: [ ``Acme Enterprises'', ``Foobar Inc.'', ``Big Corp.'' ]
\item Line 1b: [ 5000, 10000, 15000 ]
\item Line 1c: [ 400, --, 600 ]
\end{itemize}
It may appear transposed from what one ordinarily thinks of as ``lines,'' but it
corresponds better to the line numbering that tax forms tend to use. It also
makes for somewhat easier processing since the usual situation is that a form
calls for summation over a column, which can be easily obtained (e.g.,
``line[`1b', :sum]'').

A method ``add\_table\_row'' takes a hash mapping line numbers to values for a
single row, and updates each line in a manner that reflects a ``row'' being
added to a table of those lines.

\subsubsection{Special Line Names}

Generally line names in a form may be any text that has no spaces. But several
line names will receive special treatment.

\paragraph{Lines ending with an exclamation mark.} Such lines are meant for
storing metadata or informative data for a form, when there is no space for
entering the data. A common use is to create a line for transferring data to
another form. Consider, for example, a worksheet that instructs to copy line 7
to a schedule if line 7 is greater than zero, but to copy line 15 otherwise.
Instead of having to put logic into the schedule to choose between copying line
7 and line 15, the worksheet can designate a line ``fill!\@'' and the schedule
can take the value from that line.

\paragraph{Lines ending with the text ``\ldots explanation!''.} This is for
explanations that should appear on a continuation sheet. The value should be an
array where the first item is the title of the explanation and the subsequent
items are the text of the explanation, in troff format.

\paragraph{Line ``continuation!''.} The value of this line is the name of a form
found elsewhere in the output; that form will be attached as a continuation
sheet in the form of a table. (Currently, only one form can be attached to a
given form as a continuation sheet in this manner.)



\subsection{The TaxForm File Format}

As described above, data for tax computation is presented in TaxForm objects.
These objects are serialized to plain text files in the following format.
A form begins with a line starting with the word ``Form'', a space, and the
name of the form. Lines of the form follow; they must be indented with
whitespace, and the line number should be separated from the value with
whitespace. For example:
\begin{quote}
\ttfamily\obeylines\obeyspaces
Form W-2
\     first\_name          John
\     last\_name           Doe
\     a                   123-45-6789
\     b                   98-7654321
\     c                   Acme Widgets Co.
\     1                   50000
\     2                   15000
\end{quote}
Further forms may be included in the same file, delineated with the start word
``Form'' at the beginning of a line.

The line value may be any of the following:
\begin{description}
\item[blank zero] A single dash, representing a zero value.
\item[number] A number, possibly with a decimal point and possibly led with a
minus sign.
\item[date] Formatted as mm/dd/yyyy.
\item[array] A list of values, each of which is one of the above types. Arrays
may be specified in two forms: A comma-separated list surrounded by square
brackets, or another way described below.
\item[boxed data] Formatted as
$<$\emph{delim}|\emph{num}|\emph{data}$>$, where \emph{num} is a
number of boxes (usually the length of \emph{data}), \emph{delim} is a separator
for the boxes (usually a null string for splitting per character, but a social
security number would take a hyphen to fit it into a three-box unit), and
\emph{data} is a number or text string.
\item[text string] Anything else.
\end{description}

An array may be constructed by placing each array element on a separate line in
the file, replacing the line number with a double quote mark. For example:
\begin{quote}
\ttfamily\obeylines\obeyspaces
A   15
"   18
"   20
\end{quote}
is identical to ``\texttt{A [15, 18, 20]}''.

When forms are read in from a file, they are instantiated as objects of class
NamedForm, even if a class specific to the form exists. This helps to
distinguish forms read from input from those that were computed. Generally it
should be unnecessary to use any of the methods specific to particular forms on
forms read from files, since those forms have already been fully computed.

\paragraph{No Form.} If there are no forms of a certain type, the directive ``No
Form [\#]'' may be included in the import file. This avoids a warning that is
produced when the FormManager tries to find a form that is not present. (The
purpose of the warning is to avoid accidental omissions of forms.)

\paragraph{Table.} If there are many TaxForm objects of the same type (e.g.,
Charity Gift), you can save some typing by entering them in a tabular format:
\begin{quote}
\ttfamily\obeylines\frenchspacing\obeyspaces
Table Charity Gift
\    amount   cash?   name
\    500      yes     Red Cross
\    30       no      Salvation Army
\    1000     yes     NPR
\end{quote}
The last of these columns may contain values with spaces in them; the earlier
ones may not.


\subsection{The FormManager}

The FormManager maintains a complete tax return and manages the forms in that
return. Its methods mostly deal with adding, computing, and querying forms.

One interesting feature is that a FormManager has a method ``forms(name)'' that
returns an array of forms with the given name. The returned object is an Array
delegate that has an additional method ``lines'' for querying all the lines of
all the returned forms.

\paragraph{Submanagers.}
A FormManager for an entity may contain ``submanagers,'' namely
references to other FormManager objects for different entities or tax years.
This is a convenient way to draw information from a related filing.

Submanagers are added using the add\_submanager method, which requires a name
for identifying the submanager. Common names are :last\_year and :spouse. The
add\_submanager\_from\_file method creates a new FormManager, imports data from
the file, and adds a new submanager.

\paragraph{The Interviewer.} A FormManager maintains an Interviewer that allows
for TaxForms to query the user for information during execution. The Interviewer
can also store the responses to a file so that the questions need not be asked
again.

I'm generally trying to move away from Interviewer questions, instead putting
all information into informational TaxForm structures described in the Appendix.


\section{Phase One: Computation}

This section describes the operation of the first phase, computation of tax form
line items.

The usual process is as follows:
\begin{enumerate}
\item A FormManager is created and set up with the relevant parameters,
submanagers, interviewer, and so on.
\item Input forms are read using the FormManager's ``import'' method.
\item The FormManager's ``compute\_form(class)'' method is called to compute a
new form. The FormManager creates an instance of the given TaxForm class,
adds the instance to the manager, executes that instance's ``compute'' method,
and then removes the form if the ``needed?\@'' method returns false. The
``compute'' method will likely make further calls to ``compute\_form'' if
further forms are needed.
\item The computed forms are saved to a file using the FormManager's ``export''
method.
\end{enumerate}



\section{Phase Two: Form Field Marking}

The program that operates this phase is ``mark\_fields.rb''. It must be supplied
with the tax form data produced in phase 1, using the ``-i'' argument. With no
further arguments, the program lists the forms for which the fields have not
been marked yet.

To mark fields in a form, enter the form name as the first non-option argument
to the command. If the form has not been marked yet, you must provide the name
of the PDF file for the form as a second command line argument.

The program will automatically save the location of the PDF file and the
coordinates of the marked fields to a file, which by default is named
``pos-data.txt''. This file name may be changed using the ``-p'' argument.




\subsection{Boxed Data}

In some cases, the PDF form will have individual boxes for each character or
portion of text, rather than a single field for all the text. The ``boxed data''
feature enables filling in these form fields correctly.

To use this feature, the relevant TaxForm should call the method
``box\_line(line, count, split)'' where line is the line number to be placed
in boxes, count is the number of boxes available, and split is how to separate
the line's data into segments (the default is an empty string, which means each
character is separated individually). All this does is mark the relevant line in
the form as a boxed-data line; computation proceeds unchanged.

The format of boxed data in a tax form file is described above.

When the form field marking program is invoked, it will detect boxed-data lines
and treat them like arrays. Thus, each box will need to be marked individually.
However, if there are more letters than boxes available, the program will
truncate data values to fit.






\section{Phase Three: Form Filling}

To fill in the forms, create a script that creates a FormManager object with the
tax forms from phase 1 imported. Then create a MultiFormManager to handle the
filling of forms. For each form to be filled in, use the fill\_form method of
the MultiFormManager.

The MultiFormManager can automatically produce continuation sheets when there is
insufficient space. This requires providing two options:
\begin{description}
\item[continuation\_bio] Biographical text for the continuation sheets, just for
identification purposes.
\item[continuation\_display] Either :show, which will print the continuation
page to the screen; :raw, which will print the raw troff code; or :append, which
will compile the troff code and append the continuation sheet to the filled tax
form.
\end{description}





\section{Wish List}

These are features that I'd like to implement some day.

\emph{Overall conditional tests.} Right now I don't implement a variety of forms
for situations like foreign income or special rules for fishermen; I just leave
comments indicating that those are not implemented. Better would be to have some
sort of up-front survey to confirm conditions that I don't expect (basically
interview questions), and then assert those conditions in places where the
relevant functionality is not implemented.

\emph{Traceable computations.} Rather than storing result data, form lines could
store objects that maintain a tree of computations. For example, if line 7 is
the sum of lines 5 and 6, line 7 would be presented as an object referencing
those two lines and carrying an ``add'' instruction. This would help with
tracing errors and optimizing computations. It will also help with detecting
whether form or line information was never used, since that is likely an error.

\emph{Asterisks for line notes.} Currently if there is an explanatory note to a
line, it just goes in some place determined during the form marking phase;
there's no reference to the note near the line itself.

\emph{Sequence numbers.} These could most simply be implemented as instance
methods on each form class and used to order the output, though that would mean
that the sequence numbers are lost after the computation phase. They could also
be stored as a line in each form. Or they could be maintained in some external
database, though that seems unnecessarily difficult to maintain.

\emph{Generalized manager.} Rather than having to write a script that computes
the 1040, there could be a higher-level form that performs the functions of the
script in a more generalized way, among other things computing the 1040. This
has the benefit that it could compute ancillary information as a cover sheet,
such as a manifest of forms to file, the mailing address, and some summary
information about the results.

\emph{Better Interviewer questioning.} Right now, the questions themselves are
the keys for uniquely identifying the questions. This is not great for length
reasons. I'd prefer to use short names for the questions, and then have some
sort of translation table for presenting the complete question, perhaps with
help texts.

\emph{Line explanations.} I would like to be able to display explanations for
each line, to assist in reviewing computation results without having to fill in
the forms. This would require some sort of database mapping line numbers to
descriptions, either dispersed throughout the form classes or in some unified
file.




\appendix

\part*{Appendix}


\section{Additions to Forms}

\subsection{1095-B, -C}

This is the health coverage form. No information other than the items listed
below is needed.

\begin{description}
\item[hdhp?] Whether this plan was a high deductible plan that qualifies for HSA
contributions.
\item[coverage] ``individual'' or ``family''.
\item[months] The months of coverage for this plan. (Currently it is assumed
that all persons on the form are covered for the same months.) These should be
all-lowercase three-letter abbreviations of months, or ``all'' to indicate all
months.
\end{description}

\subsection{1098}

\begin{description}
\item[property] The name of the Real Estate form with which this 1098
corresponds.
\item[balance] The average mortgage balance on this loan. See Pub.\ 936.
You can compute this by averaging the balances on January 1 and December 31, or
you can multiply the interest paid by the lowest interest rate of the year.
\end{description}

\subsection{1099-DIV}

\begin{description}
\item[qexception?] If this form shows qualified dividends, this flag
must be set to indicate whether an exception applies. See 1040 instructions for
line 3a. The exceptions relate to unusual transactions.
\end{description}


\subsection{1099-G}

\begin{description}
\item[payer] The name of the payer (usually the state)
\end{description}


\subsection{1099-R}

\begin{description}
\item[2b.not\_determined?] The relevant checkbox under line 2b.
\item[2b.total\_distribution?] The relevant checkbox under line 2b.
\item[ira-sep-simple?] Whether the checkbox on the form with this name is
checked.
\item[destination] Where this distribution went. It can be ``Roth conversion''
(currently the only one supported).
\end{description}


\section{Informational Forms}



\subsection{Alimony}

Alimony received. See Pub.\ 504.

\begin{description}
\item[amount] The amount of alimony received.
\end{description}


\subsection{Asset}

A form for a capitalized asset.

\begin{description}
\item[date] The date the asset was put into service.
\item[amount] The dollar value of the asset when purchased.
\item[179?] Whether the asset is a section 179 deductible asset. Generally,
tangible personal property and computer software qualify. See Pub.~946 chapter
2.
\item[listed?] Whether the asset is a listed asset, see Pub.~946 chapter 5.
\item[dc\_category] The depreciation category for the asset in DC, see the FP-31
instructions, under Depreciation Guidelines.
\item[dc\_type] Choose from ``reference'', ``fixed'', or ``other'' to categorize
the asset for DC Form FP-31, lines 1--3.
\item[description] A description of the asset.
\end{description}


\subsection{Biographical}

Biographical information for an individual tax filer. Spouses should have a
separate Biographical form.

\begin{description}
\item[whose] The individual to whom this form pertains. May be ``mine'' or
``spouse''.
\item[first\_name] The first name and middle initial.
\item[last\_name] The last name.
\item[phone] Telephone number.
\item[ssn] Social Security number, with dashes.
\item[home\_address] The first line of the home address.
\item[apt\_no] The apartment number for the address.
\item[city\_zip] The city, state, and ZIP code.
\item[birthday] The person's birthday.
\item[blind?] Whether the person is blind.
\item[occupation] Your occupation
\end{description}

\subsection{Business Expense}

A deductible business expense. See IRS Pub.~535, chapter 11.

\begin{description}
\item[date] The date of the business expense.
\item[amount] The dollar amount of the expense.
\item[category] The category of expense. ``Meals'' and ``Utilities'' will
automatically be halved, the former per the IRS 50\% rule and the latter on the
assumption that 50\% of the expense was for non-business purposes. Other common
values are ``Supplies'', ``Travel'', and ``Membership''.
\item[description] A description of the expense.
\end{description}


\subsection{Charity Gift}

A charitable contribution.

\begin{description}
\item[amount] The dollar value of the contribution.
\item[cash?] Whether the contribution was in cash (as opposed to in-kind).
\item[name] The name of the charity.
\item[documented?] Whether the donation was documented. See Pub.\ 526. This is
checked if the contribution is over the \$250 threshold. Currently it is assumed
that any documentation meets the requirements of a contemporaneous written
acknowledgment (received before filing the return).
\end{description}
For charity gifts that are non-cash that total over \$500, Form 8283 must also
be filled, which requires additional information on the nature of the gift:
\begin{description}
\item[vin] The VIN of a donated vehicle.
\item[description] A description of the donated goods and their condition.
Clothing must be donated in good used condition.
\item[date] The date the contribution was made.
\item[date\_acq] The approximate date on which the property was acquired, or
``Various'' if the items were acquired on various dates at least 12 months ago.
\item[how\_acq] How the property was acquired, for example by purchase, gift,
inheritance, or exchange.
\item[basis] The basis or cost of the property.
\item[method] Method for determining fair market value. IRS examples are
appraisal, thrift shop value, catalog, or comparable sales.
\end{description}


\subsection{Dependent}

A dependent.

\begin{description}
\item[name] The dependent's name.
\item[ssn] The dependent's social security number.
\item[relationship] The dependent's relationship with the form provider.
\item[qualifying] Whether the dependent qualifies for a tax credit. Options are
``child'' for the child tax credit, ``other'' for the credit for other
dependents, and ``none'' where the dependent is not qualifying.
\end{description}


\subsection{Estimated Tax}

Estimated tax payment made to the IRS.

\begin{description}
\item[amount] The amount of estimated tax paid.
\end{description}

\subsection{HSA Contribution}

Records of contributions to an HSA account.

\begin{description}
\item[ssn] The SSN of the beneficiary of the HSA account.
\item[contributions] Contributions made to the HSA, but not including employer
contributions or HSA/Archer MSA rollovers or HSA funding distributions.
\end{description}

\subsection{Home Office}

A portion of a home used as an office for a particular business.

\begin{description}
\item[type] The type of business associated with this home office. Currently the
only supported type is ``partnership''.
\item[ein] The EIN of the relevant business.
\item[method] Either ``simplified'' or ``actual''.
\item[sqft] Square footage of the home office area.
\item[daycare?] Whether the relevant business was a daycare.
\item[property] The name of the Real Estate form with which this home office is
associated.
\end{description}

\subsection{State Estimated Tax}

Estimated tax payment made to a state.

\begin{description}
\item[amount] The amount of estimated tax paid.
\item[state] The two-letter code for the state.
\end{description}




\subsection{State Tax}

A record of state taxes paid in the relevant tax year (usually for the previous
year). This is the amount actually paid to the state tax office, (i.e., the
amount due), not the total tax including what was already paid via withholdings.

\begin{description}
\item[amount] The amount of tax paid.
\item[name] A description (not used for computation).
\end{description}

\subsection{Traditional IRA Contribution}

A contribution to a traditional IRA.

\begin{description}
\item[amount] The amount contributed.
\item[tax\_year] The tax year of the contribution.
\item[date] The date of the contribution.
\end{description}


\subsection{Partner}

A single partner within a partnership.

\begin{description}
\item[name] The partner's name.
\item[liability] Either ``general'' or ``limited''.
\item[nationality] Either ``domestic'' or ``foreign''.
\item[type] The entity type of the partner. Currently only ``Individual'' is
supported fully. Other possibilities include ``Estate''. (This should also, in
the future, include corporations, partnerships, trusts, nonprofits, and foreign
governments, among others.)
\item[share] This partner's share of profits and losses, as a decimal fraction
such that all the partners' shares add to 1. Currently it is assumed that shares
of profits and shares of losses are equal.
\item[capital] This partner's percentage contribution of capital to the
partnership, as a decimal fraction such that all the partners' shares add to 1.
\item[ssn] The SSN of the partner.
\item[country] The country of citizenship of the partner, relevant to Form 1065
Schedule B-1, part II, column iii.
\item[address] The first line of the partner's address.
\item[address2] The second line of the partner's address.
\end{description}




\subsection{Partnership}

Biographical information for a partnership.

\begin{description}
\item[name] The name of the partnership.
\item[address] The first line of the address.
\item[address2] The second line of the address.
\item[business] The line of business of the partnership, Form 1065 line A.
\item[product] The product or service of the partnership, Form 1065 line B.
\item[code] The business code, Form 1065 line C.
\item[ein] The employer identification number of the partnership.
\item[start] The start date of the partnership.
\item[accounting] ``Cash'', ``Accrual'', or other accounting method.
\end{description}

\subsection{Real Estate}

A description of real property, used in a variety of contexts and to link other
forms that deal with a particular property.

\begin{description}
\item[property] A name for the property that will uniquely identify the property
across tax forms.
\item[basis] The basis price for the property, i.e., the purchase value. See
Pub.\ 551.
\item[sqft] The square footage of the property.
\item[purchase\_date] The date on which the property was purchased.
\end{description}




\end{document}
