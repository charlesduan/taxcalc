\documentclass[12pt]{article}

\title{Framework for Tax Preparation}
\author{Charles Duan}

\begin{document}

\maketitle

This is a software system for automating the computation and filling of tax
forms. Much like commercial software for tax preparation, the object of the
system is to allow a user to enter data from received tax forms and other
sources, and then answer interview questions, to produce completed tax forms
ready for filing.

Unlike commercial tax software, however, this system is not intended to
comprehensively cover all tax forms and options. Instead, a second object of
this system is to provide a framework to make it reasonably easy for
programming-adept users to add tax forms as necessary to their personal
situations and needs.

\section{Overview of Operation}

Use of the system proceeds in three phases.

The first phase is the computation of form values. The user enters relevant data
from received tax forms and other sources into files, and then writes a short
script that imports that data and computes the values for tax forms to be
prepared. This produces a plain text file containing the line item data to be
filled into the tax forms.

The second phase is the identification of form fields. For each tax form to be
filled in, the user provides the PDF file for the blank form and also the file
of line item data. A program displays a graphical interface where the user can
click on the blank spaces where each line item should be entered. The result of
this is a second plain text data file that lists the coordinates of each of the
form fields.

The third phase is the actual filling of the forms. The user writes another
short script that receives the data files from the first and second phases, and
places the line items from the first-phase file into the PDF form. If any
changes need to be made subsequently, the user may simply rerun the first-phase
computation script and then rerun the third-phase filling script, thereby
updating all of the tax forms automatically.


\section{Phase One: Computation}

This section describes the operation of the first phase, computation of tax form
line items.

\subsection{Structure of Tax Form Data}

The basic data structure used throughout the system is the TaxForm, which in
essence is simply a named table mapping line numbers to values.



\subsection{Form Submanagers}

The main FormManager for an entity may contain ``submanagers,'' namely
references to other FormManager objects for different entities or tax years.
This is a convenient way to draw information from a related filing.

Submanagers are added using the add\_submanager method, which requires a name
for identifying the submanager. Common names are :last\_year and :spouse. The
add\_submanager\_from\_file method creates a new FormManager, imports data from
the file, and adds a new submanager.


\section{Phase Two: Form Field Marking}

The program that operates this phase is ``mark\_fields.rb''. It must be supplied
with the tax form data produced in phase 1, using the ``-i'' argument. With no
further arguments, the program lists the forms for which the fields have not
been marked yet.

To mark fields in a form, enter the form name as the first non-option argument
to the command. If the form has not been marked yet, you must provide the name
of the PDF file for the form as a second command line argument.

The program will automatically save the location of the PDF file and the
coordinates of the marked fields to a file, which by default is named
``pos-data.txt''. This file name may be changed using the ``-p'' argument.



\section{Phase Three: Form Filling}

To fill in the forms, create a script that creates a FormManager object with the
tax forms from phase 1 imported. Then create a MultiFormManager to handle the
filling of forms. For each form to be filled in, use the fill\_form method of
the MultiFormManager.

The MultiFormManager can automatically produce continuation sheets when there is
insufficient space. This requires providing two options:
\begin{description}
\item[continuation\_bio] Biographical text for the continuation sheets, just for
identification purposes.
\item[continuation\_display] Either :show, which will print the continuation
page to the screen; :raw, which will print the raw troff code; or :append, which
will compile the troff code and append the continuation sheet to the filled tax
form.
\end{description}



\appendix


\section{Additions to Forms}

\subsection{1098}

\begin{description}
\item[property] The name of the Real Estate form with which this 1098
corresponds.
\item[balance] The average mortgage balance on this loan. See Pub.\ 936.
You can compute this by averaging the balances on January 1 and December 31, or
you can multiply the interest paid by the lowest interest rate of the year.
\end{description}

\subsection{1099-DIV}

\begin{description}
\item[qualified\_exception] If this form shows qualified dividends, this flag
must be set to indicate whether an exception applies. See 1040 instructions.
\end{description}


\subsection{1099-R}

\begin{description}
\item[2b.not\_determined?] The relevant checkbox under line 2b.
\item[2b.total\_distribution?] The relevant checkbox under line 2b.
\item[ira-sep-simple?] Whether the checkbox on the form with this name is
checked.
\item[destination] Where this distribution went. It can be ``Roth conversion''
(currently the only one supported).
\end{description}


\section{Forms}



\subsection{Alimony}

Alimony received. See Pub.\ 504.

\begin{description}
\item[amount] The amount of alimony received.
\end{description}


\subsection{Asset}

A form for a capitalized asset.

\begin{description}
\item[date] The date the asset was put into service.
\item[amount] The dollar value of the asset when purchased.
\item[179?] Whether the asset is a section 179 deductible asset.
\item[listed?] Whether the asset is a listed asset, see Pub.~946 chapter 5.
\item[dc\_category] The depreciation category for the asset in DC, see the FP-31
instructions, under Depreciation Guidelines.
\item[dc\_type] Choose from ``reference'', ``fixed'', or ``other'' to categorize
the asset for DC Form FP-31, lines 1--3.
\item[description] A description of the asset.
\end{description}





\subsection{Business Expense}

A deductible business expense. See IRS Pub.~535, chapter 11.

\begin{description}
\item[date] The date of the business expense.
\item[amount] The dollar amount of the expense.
\item[category] The category of expense. ``Meals'' and ``Utilities'' will
automatically be halved, the former per the IRS 50\% rule and the latter on the
assumption that 50\% of the expense was for non-business purposes. Other common
values are ``Supplies'', ``Travel'', and ``Membership''.
\item[description] A description of the expense.
\end{description}


\subsection{Charity Gift}

A charitable contribution.

\begin{description}
\item[amount] The dollar value of the contribution.
\item[cash?] Whether the contribution was in cash (as opposed to in-kind).
\item[name] The name of the charity.
\item[documented?] Whether the donation was documented. See Pub.\ 526. This is
checked if the contribution is over the \$250 threshold.
\end{description}



\subsection{Dependent}

A dependent.

\begin{description}
\item[name] The dependent's name.
\item[ssn] The dependent's social security number.
\item[relationship] The dependent's relationship with the form provider.
\item[qualifying] Whether the dependent qualifies for a tax credit. Options are
``child'' for the child tax credit, ``other'' for the credit for other
dependents, and ``none'' where the dependent is not qualifying.
\end{description}

\subsection{Home Office}

A portion of a home used as an office for a particular business.

\begin{description}
\item[type] The type of business associated with this home office. Currently the
only supported type is ``partnership''.
\item[ein] The EIN of the relevant business.
\item[method] Either ``simplified'' or ``actual''.
\item[sqft] Square footage of the home office area.
\item[daycare?] Whether the relevant business was a daycare.
\item[property] The name of the Real Estate form with which this home office is
associated.
\end{description}

\subsection{Traditional IRA Contribution}

A contribution to a traditional IRA.

\begin{description}
\item[amount] The amount contributed.
\item[tax\_year] The tax year of the contribution.
\item[date] The date of the contribution.
\end{description}


\subsection{Partner}

A single partner within a partnership.

\begin{description}
\item[name] The partner's name.
\item[liability] Either ``general'' or ``limited''.
\item[nationality] Either ``domestic'' or ``foreign''.
\item[type] The entity type of the partner. Currently only ``Individual'' is
supported fully. Other possibilities include ``Estate''. (This should also, in
the future, include corporations, partnerships, trusts, nonprofits, and foreign
governments, among others.)
\item[share] This partner's share of profits and losses, as a decimal fraction
such that all the partners' shares add to 1. Currently it is assumed that shares
of profits and shares of losses are equal.
\item[capital] This partner's percentage contribution of capital to the
partnership, as a decimal fraction such that all the partners' shares add to 1.
\item[ssn] The SSN of the partner.
\item[country] The country of citizenship of the partner, relevant to Form 1065
Schedule B-1, part II, column iii.
\item[address] The first line of the partner's address.
\item[address2] The second line of the partner's address.
\end{description}




\subsection{Partnership}

Biographical information for a partnership.

\begin{description}
\item[name] The name of the partnership.
\item[address] The first line of the address.
\item[address2] The second line of the address.
\item[business] The line of business of the partnership, Form 1065 line A.
\item[product] The product or service of the partnership, Form 1065 line B.
\item[code] The business code, Form 1065 line C.
\item[ein] The employer identification number of the partnership.
\item[start] The start date of the partnership.
\item[accounting] ``Cash'', ``Accrual'', or other accounting method.
\end{description}

\subsection{Real Estate}

A description of real property, used in a variety of contexts and to link other
forms that deal with a particular property.

\begin{description}
\item[property] A name for the property that will uniquely identify the property
across tax forms.
\item[basis] The basis price for the property, i.e., the purchase value. See
Pub.\ 551.
\item[sqft] The square footage of the property.
\item[purchase\_date] The date on which the property was purchased.
\end{description}




\end{document}
