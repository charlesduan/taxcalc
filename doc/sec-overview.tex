\section{Introduction}

This is a system for preparing tax forms. The idea is to translate the relevant
computations and forms into modular computer programs that, when assembled
together, can compute a complete tax return based on received tax information.

The basic motivation for this system is the separation of form and content. Tax
forms are essentially association tables, in which numbered lines are associated
with blank spaces to be filled with numbers or other data. The content could be
represented as a two-column spreadsheet, but the complexity of calculations and
interactions among multiple forms makes an object-oriented representation
in a general-purpose programming language preferable. Thus, the system separates
the computations necessary to figure the tax forms first, and then enters the
figured numbers into the graphical forms.

Several advantages arise out of automating the filling of tax forms in this way.
Automating the computations allows for testing out different elections during
the tax computation (e.g., opting for itemized deductions or the standard
deduction) without having to recompute the numbers by hand. It allows for
programming accuracy checks to find errors in form inputs (e.g., missed lines on
W-2 forms). It also saves time across years, since the computations for most tax
forms do not change greatly across years. And it enables multiple people's taxes
to be computed using the same software. The primary disadvantage is that
programming general-case computations is less efficient than computing a single
tax return, but it is believed that the aforementioned advantages outweigh the
time spent.

Use of the system proceeds in three phases.

The first phase is the computation of form values. The user enters relevant data
from received tax forms and other sources into files, and then writes a short
script that imports that data and computes the values for tax forms to be
prepared. This produces a plain text file containing the line item data to be
filled into the tax forms.

The second phase is the identification of form fields. For each tax form to be
filled in, the user provides the PDF file for the blank form and also the file
of line item data. A program displays a graphical interface where the user can
click on the blank spaces where each line item should be entered. The result of
this is a second plain text data file that lists the coordinates of each of the
form fields.

The third phase is the actual filling of the forms. This now involves simply
running a script that does the form filling.




