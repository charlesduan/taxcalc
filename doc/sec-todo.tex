\section{Wish List}

These are features that I'd like to implement some day.

\emph{Traceable computations.} Rather than storing result data, form lines could
store objects that maintain a tree of computations. For example, if line 7 is
the sum of lines 5 and 6, line 7 would be presented as an object referencing
those two lines and carrying an ``add'' instruction. This would help with
tracing errors and optimizing computations. It will also help with detecting
whether form or line information was never used, since that is likely an error.

\emph{Sequence numbers.} These could most simply be implemented as instance
methods on each form class and used to order the output, though that would mean
that the sequence numbers are lost after the computation phase. They could also
be stored as a line in each form. Or they could be maintained in some external
database, though that seems unnecessarily difficult to maintain.

\emph{Generalized manager.} Rather than having to write a script that computes
the 1040, there could be a higher-level form that performs the functions of the
script in a more generalized way, among other things computing the 1040. This
has the benefit that it could compute ancillary information as a cover sheet,
such as a manifest of forms to file, the mailing address, and some summary
information about the results.

\emph{Better Interviewer questioning.} Right now, the questions themselves are
the keys for uniquely identifying the questions. This is not great for length
reasons. I'd prefer to use short names for the questions, and then have some
sort of translation table for presenting the complete question, perhaps with
help texts.

\emph{Line explanations.} I would like to be able to display explanations for
each line, to assist in reviewing computation results without having to fill in
the forms. This would require some sort of database mapping line numbers to
descriptions, either dispersed throughout the form classes or in some unified
file.

\emph{Forced line aliases.} It is possible to make an alias for a line (e.g.,
``agi'' for the relevant line of the 1040), which ensures that other forms
remain correct even if the number of that line changes. Currently, a form may
reference another form's lines by line number or alias, but the system will
issue a warning if an aliased line is reference by number. It would be better if
every reference to an external form's lines were by alias rather than number.

%
% DONE
% \emph{Assertions without requiring input.} Many forms use an interview
% question to confirm that the user does not meet a rare special condition.
% Better would be for the forms to dump all these assertions into a file for the
% user to review later.

\emph{Rename managers.} The QBIManager class, for example---it would better be
named QBIAnalysis.

%
% DONE
%
% \emph{Name registration.} Forms effectively have two names, a class name and a
% textual name. Reconciling these into a single name would be helpful.

\emph{Push instruction testing.} If a form produces a line value that is meant
to be incorporated into another form, it would be helpful to have some sort of
assertion that the line is indeed incorporated into that other form. See the
section on Push versus Pull Filling.
