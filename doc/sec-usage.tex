\section{Recommended Procedure for Annual Filing}

This section provides recommendations on preparing an annual tax filing. This
includes updating the software in view of changes to the tax laws and forms, as
well as preparing the returns themselves.

1.\quad Prepare the script for running the computations, and enter the data for
received forms (W-2s, 1099s, etc.). It is preferable to do this first rather
than updating the tax form computations, because it allows for partially running
the program in advance. This has two benefits as you update the computations: It
reveals bugs earlier, and it lets you assess with real data whether it is worth
implementing a new computation.

At this step, be sure to review the informational forms in the Appendix and
ensure that you have entered all relevant information required there.

2.\quad Update the computations. Start with the main form being computed (e.g.,
Form 1040), and work line by line through the script. I found it preferable to
address each form dependency as it came immediately, resulting in a depth-first
update. That is, when the 1040 calls for a result from Schedule B, I would pause
work on the 1040 and go update Schedule B. The advantage of this is that you can
continue running partial tests of the computations. The main disadvantage is
that it becomes hard to keep track of which form you are working on; I generally
addressed this by leaving comments in the files as I worked on them.

Once all the computations are done and run without errors, move on to phase 2
and mark the form fields. I found it useful to mark both the forms and the
worksheets, even though the worksheets do not need to be filed, because it made
it easier to review everything for accuracy.

3.\quad Run the script that fills in all the forms (phase 3).
Once you have reviewed the draft return and corrected any issues, you can
optimize the return (e.g., try different filing statuses, move dependents
around, choose standard versus itemized deductions, and so on). Once you are
satisfied, you can print, sign, and file your returns.


