\section{Object Classes and Data Structures}

The tax computation system is written in Ruby and is organized around two main
classes: the TaxForm and the FormManager.


\subsection{The TaxForm API}

Generally, each tax form is represented as an instance of a Ruby class for that
form, a subclass of TaxForm. A TaxForm has four essential methods:
\begin{description}
\item[name] The name of this form. The word ``Form'' should not precede names.
\item[year] The tax year for which this form has been updated. This is used to
check whether the form has been updated yet. (An out-of-date form will still
be executed; it will just produce a warning.)
\item[compute] Compute the form. This method should fill in all the lines of the
form.
\item[needed?] Whether the tax form is ultimately needed for inclusion in the
return. This method should be called after compute, and it should inspect the
results of the lines of the form to determine if the form needs to be filed.
\end{description}

A form is essentially a hash table that maps line numbers to data values. Line
numbers may be any strings so long as they contain no spaces. Values may be
atomic values (numbers, strings, dates), or may be arrays of those atomic
values. The order of lines is remembered (and can be altered manually).


\subsubsection{Referring to Lines in a Form}

The TaxForm has several methods with the name ``line'' for getting and setting
lines:
\begin{itemize}
\item line[\#] returns the value of that line. If the value is an array or is
unset, an error is raised.
\item line[\#, :opt] returns the value of the line, or if the line is
unset it returns BlankZero.
\item line[\#, :all] returns an array for this line's values. It always returns
an array, even if the line only has an atomic value.
\item line[\#, :sum] returns the sum of this line's values.
\item line[\#, :present] returns true or false depending on whether the line is
set.
\item line[\#] = x will set the line to an atomic value; it will raise an error
if given an array. If the line is already set, a warning will be raised.
\item line[\#, :all] = [ x, y, z ] will set the line to an array value.
\item line[\#, :add] = x will force the line to be an array and then will append
the given value to that array.
\item line[\#, :overwrite] = x will set the line to an atomic value, and will
not warn if the line was already set.
\end{itemize}
As a shortcut, the ``method\_missing'' method for TaxForm will infer line
numbers from methods named ``line\_\#''.

Every TaxForm is associated with a FormManager. Many of the methods of TaxForm
are delegated to its FormManager as a matter of convenience.

\subsubsection{Line Name Aliases}

A line may be given multiple ``names,'' or aliases, separated with a slash. Each
portion separated by slashes becomes an alias for the same line. For example, a
line named ``8b/agi'' would have two aliases ``8b'' and ``agi'' that could both
be used to refer to the same line.

The purpose of aliases is to provide a stable name for key lines in forms, when
the numbering may change across years. As a result, the non-numeric alias is
preferable for use as a line name, especially when referring to the line from
different forms. A warning will be given when the numeric alias (assumed to be
the first alias from the line's full name) is used alone in another form.

\subsubsection{How Tables Are Stored in Forms}

Consider a form that requires input like so:
\begin{center}
\begin{tabular}{|l|l|l|}
\hline
\textbf{Line 1a} & \textbf{1b} & \textbf{1c} \\
\textbf{Business name} & \textbf{Income} & \textbf{Expenses} \\
\hline
Acme Enterprises & 5000 & 400 \\
\hline
Foobar Inc. & 10000 & \\
\hline
Big Corp. & 15000 & 600 \\
\hline
\end{tabular}
\end{center}
This would be stored in a TaxForm as three lines named 1a, 1b, and 1c as
follows:
\begin{itemize}
\item Line 1a: [ ``Acme Enterprises'', ``Foobar Inc.'', ``Big Corp.'' ]
\item Line 1b: [ 5000, 10000, 15000 ]
\item Line 1c: [ 400, --, 600 ]
\end{itemize}
It may appear transposed from what one ordinarily thinks of as ``lines,'' but it
corresponds better to the line numbering that tax forms tend to use. It also
makes for somewhat easier processing since the usual situation is that a form
calls for summation over a column, which can be easily obtained (e.g.,
``line[`1b', :sum]'').

A method ``add\_table\_row'' takes a hash mapping line numbers to values for a
single row, and updates each line in a manner that reflects a ``row'' being
added to a table of those lines.

\subsubsection{Special Line Names}

Generally line names in a form may be any text that has no spaces. But several
line names will receive special treatment.

\paragraph{Lines ending with an exclamation mark.} Such lines are meant for
storing metadata or informative data for a form, when there is no space for
entering the data. A common use is to create a line for transferring data to
another form. Consider, for example, a worksheet that instructs to copy line 7
to a schedule if line 7 is greater than zero, but to copy line 15 otherwise.
Instead of having to put logic into the schedule to choose between copying line
7 and line 15, the worksheet can designate a line ``fill!\@'' and the schedule
can take the value from that line.

\paragraph{Lines ending with the text ``\ldots explanation!''.} This is for
explanations that should appear on a continuation sheet. The value should be an
array where the first item is the title of the explanation and the subsequent
items are the text of the explanation, in troff format.

\paragraph{Line ``continuation!''.} The value of this line is the name of a form
found elsewhere in the output; that form will be attached as a continuation
sheet in the form of a table. (Currently, only one form can be attached to a
given form as a continuation sheet in this manner.)

\paragraph{Lines ending with ``*note''.} These lines indicate a footnote to a
line (e.g., line 15*note would be a footnote to line 15). When the form is
filled in, a footnote mark (currently one or more asterisks) will be placed
after the value in the specified line and in front of the value of the line's
note.



\subsection{The TaxForm File Format}

As described above, data for tax computation is presented in TaxForm objects.
These objects are serialized to plain text files in the following format.
A form begins with a line starting with the word ``Form'', a space, and the
name of the form. Lines of the form follow; they must be indented with
whitespace, and the line number should be separated from the value with
whitespace. For example:
\begin{quote}
\ttfamily\obeylines\obeyspaces
Form W-2
\     first\_name          John
\     last\_name           Doe
\     a                   123-45-6789
\     b                   98-7654321
\     c                   Acme Widgets Co.
\     1                   50000
\     2                   15000
\end{quote}
Further forms may be included in the same file, delineated with the start word
``Form'' at the beginning of a line.

The line value may be any of the following:
\begin{description}
\item[blank zero] A single dash, representing a zero value.
\item[number] A number, possibly with a decimal point and possibly led with a
minus sign.
\item[date] Formatted as mm/dd/yyyy.
\item[array] A list of values, each of which is one of the above types. Arrays
may be specified in two forms: A comma-separated list surrounded by square
brackets, or another way described below.
\item[boxed data] Formatted as
$<$\emph{delim}|\emph{num}|\emph{data}$>$, where \emph{num} is a
number of boxes (usually the length of \emph{data}), \emph{delim} is a separator
for the boxes (usually a null string for splitting per character, but a social
security number would take a hyphen to fit it into a three-box unit), and
\emph{data} is a number or text string.
\item[text string] Anything else.
\end{description}

An array may be constructed by placing each array element on a separate line in
the file, replacing the line number with a double quote mark. For example:
\begin{quote}
\ttfamily\obeylines\obeyspaces
A   15
"   18
"   20
\end{quote}
is identical to ``\texttt{A [15, 18, 20]}''.

When forms are read in from a file, they are instantiated as objects of class
NamedForm, even if a class specific to the form exists. This helps to
distinguish forms read from input from those that were computed. Generally it
should be unnecessary to use any of the methods specific to particular forms on
forms read from files, since those forms have already been fully computed.

\paragraph{No Form.} If there are no forms of a certain type, the directive ``No
Form [\#]'' may be included in the import file. This avoids a warning that is
produced when the FormManager tries to find a form that is not present. (The
purpose of the warning is to avoid accidental omissions of forms.)

\paragraph{Table.} If there are many TaxForm objects of the same type (e.g.,
Charity Gift), you can save some typing by entering them in a tabular format:
\begin{quote}
\ttfamily\obeylines\frenchspacing\obeyspaces
Table Charity Gift
\    amount   cash?   name
\    500      yes     Red Cross
\    30       no      Salvation Army
\    1000     yes     NPR
\end{quote}
The last of these columns may contain values with spaces in them; the earlier
ones may not.


\subsection{The FormManager}

The FormManager maintains a complete tax return and manages the forms in that
return. Its methods mostly deal with adding, computing, and querying forms.

One interesting feature is that a FormManager has a method ``forms(name)'' that
returns an array of forms with the given name. The returned object is an Array
delegate that has an additional method ``lines'' for querying all the lines of
all the returned forms.

\paragraph{Submanagers.}
A FormManager for an entity may contain ``submanagers,'' namely
references to other FormManager objects for different entities or tax years.
This is a convenient way to draw information from a related filing.

Submanagers are added using the add\_submanager method, which requires a name
for identifying the submanager. Common names are :last\_year and :spouse. The
add\_submanager\_from\_file method creates a new FormManager, imports data from
the file, and adds a new submanager.

\paragraph{The Interviewer.} A FormManager maintains an Interviewer that allows
for TaxForms to query the user for information during execution. The Interviewer
can also store the responses to a file so that the questions need not be asked
again.

I'm generally trying to move away from Interviewer questions, instead putting
all information into informational TaxForm structures described in the Appendix.



