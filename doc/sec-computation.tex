\section{Phase One: Computation}

This section describes the operation of the first phase, computation of tax form
line items.

The usual process is as follows:
\begin{enumerate}
\item A FormManager is created and set up with the relevant parameters,
submanagers, interviewer, and so on.
\item Input forms are read using the FormManager's ``import'' method.
\item The FormManager's ``compute\_form(class)'' method is called to compute a
new form. The FormManager creates an instance of the given TaxForm class,
adds the instance to the manager, executes that instance's ``compute'' method,
and then removes the form if the ``needed?\@'' method returns false. The
``compute'' method will likely make further calls to ``compute\_form'' if
further forms are needed.
\item The computed forms are saved to a file using the FormManager's ``export''
method.
\end{enumerate}


\subsection{When Forms Are Computed}

Computations of forms are always invoked by either the main program or another
form in the course of that other form's computation. This appears to mimic the
IRS's expectations that one will start filling Form 1040 line by line, and
figure each schedule and other form as the need arises. In particular, many
forms must be computed after some portion of the 1040 is already completed,
meaning that they must be invoked in the midst of the 1040 computation.

In some cases, notably the computation of nondeductible IRA contributions, the
IRS's instructions are problematically nonlinear and even create circular
dependencies. It is useful though unfortunate that the rigorous computations of
this system expose these issues.



\subsection{Push versus Pull Filling}

The IRS instructions often use a ``push''-style approach in which a worksheet or
form is filled to reach a key number, which is to be then entered or included in
another form. For example, the last line of Schedule A is to be entered onto the
relevant line of Form 1040.

This program instead follows a ``pull'' convention in which each form is
responsible for gathering information from all relevant forms. In the above
example, the code for Schedule A would simply compute its last line without
using it, and Form 1040 will query the completed Schedule A to find the
last-line value to fill. This convention is consistent with the expectation that
each form is responsible for performing all computations necessary to fill the
form, which includes invoking computations of other necessary forms. The
implementer of the 1040 form would know better whether the deduction line should
be filled with the Schedule A value (as opposed to, say, the standard
deduction), so it would be problematic for Schedule A to insert the value onto
the 1040 prematurely.

A disadvantage of the ``pull'' approach is that many lines are computed from
multiple and possibly unexpected forms. The tax line on the 1040, for example,
includes several different taxes, and the 1040 needs to combine all those
values, such as the Net Investment Income Tax. A helpful feature that could be
implemented in the future would be to allow one form, such as the NIIT form, to
add an assertion to an external form to insure that its values are used.



\subsection{Preventing Errors}

A key problem for programming tax computation is ensuring accuracy. The large
amount of input data is a likely source of errors---mistyped numbers or missed
forms---and many unusual rules can arise that require special edge-case
handling. A few conventions are used to minimize errors.

\emph{Assertions.} In many cases, certain conditions are so rare that they are
not worth implementing. The computation programs will thus raise assertions if
they detect conditions that are not implemented. In some cases, detecting the
condition will not be possible on the data alone, in which an interview question
is asked to determine that the special condition is not present.

\emph{Form checking.} For input forms, it is assumed that the user may fail to
enter certain ones. As a result, when a computation uses an input form that is
not present, a warning is given unless the user has explicitly indicated that
there are no forms of that type (using the ``No Form'' instruction described
previously). This checking could be improved, because currently there is not
much distinction between input forms (1099s, W-2s) and computation forms
(Schedule A, Form 8606).

\emph{Line presence.} If a computation requests a particular line, the line must
be present unless the ``:opt'' option is given as described above, and it must
not be an array unless an array was expected.



