\section{Phase Two: Form Field Marking}

The program that operates this phase is ``mark\_fields.rb''. It must be supplied
with the tax form data produced in phase 1, using the ``-i'' argument. With no
further arguments, the program lists the forms for which the fields have not
been marked yet.

To mark fields in a form, enter the form name as the first non-option argument
to the command. If the form has not been marked yet, you must provide the name
of the PDF file for the form as a second command line argument.

The program will automatically save the location of the PDF file and the
coordinates of the marked fields to a file, which by default is named
``pos-data.txt''. This file name may be changed using the ``-p'' argument.




\subsection{Boxed Data}

In some cases, the PDF form will have individual boxes for each character or
portion of text, rather than a single field for all the text. The ``boxed data''
feature enables filling in these form fields correctly.

To use this feature, the relevant TaxForm should call the method
``box\_line(line, count, split)'' where line is the line number to be placed
in boxes, count is the number of boxes available, and split is how to separate
the line's data into segments (the default is an empty string, which means each
character is separated individually). All this does is mark the relevant line in
the form as a boxed-data line; computation proceeds unchanged.

The format of boxed data in a tax form file is described above.

When the form field marking program is invoked, it will detect boxed-data lines
and treat them like arrays. Thus, each box will need to be marked individually.
However, if there are more letters than boxes available, the program will
truncate data values to fit.





