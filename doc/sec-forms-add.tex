\section{Additions to Forms}

\subsection{1095-B, -C}

This is the health coverage form. No information other than the items listed
below is needed.

\begin{description}
\item[hdhp?] Whether this plan was a high deductible plan that qualifies for HSA
contributions.
\item[coverage] ``individual'' or ``family''.
\item[months] The months of coverage for this plan. (Currently it is assumed
that all persons on the form are covered for the same months.) These should be
all-lowercase three-letter abbreviations of months, or ``all'' to indicate all
months.
\end{description}

\subsection{1098}

\begin{description}
\item[property] The name of the Real Estate form with which this 1098
corresponds.
\item[balance] The average mortgage balance on this loan. See Pub.\ 936.
You can compute this by averaging the balances on January 1 and December 31, or
you can multiply the interest paid by the lowest interest rate of the year.
\end{description}

\subsection{1099-DIV}

\begin{description}
\item[qexception?] If this form shows qualified dividends, this flag
must be set to indicate whether an exception applies. See 1040 instructions for
line 3a. The exceptions relate to unusual transactions.
\end{description}


\subsection{1099-G}

\begin{description}
\item[payer] The name of the payer (usually the state)
\end{description}


\subsection{1099-R}

\begin{description}
\item[2b.not\_determined?] The relevant checkbox under line 2b.
\item[2b.total\_distribution?] The relevant checkbox under line 2b.
\item[ira-sep-simple?] Whether the checkbox on the form with this name is
checked.
\item[destination] Where this distribution went. It can be ``roth''
(currently the only one supported).
\end{description}



