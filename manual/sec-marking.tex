\section{Phase Two: Form Field Marking}

The program that operates this phase is ``ui/mark.rb''. See the options help on
running that script for details.

The script is provided with the following inputs:
\begin{itemize}
\item The computed TaxForm file(s), which are used to discern the lines for each
form.
\item The name of the form to work on.
\item The location of a PDF file. The script can predict default URLs for
several IRS forms and attempt to download them.
\item Which pages of the PDF file to include, useful for worksheets contained
within instructions.
\end{itemize}
The program will automatically save the location of the PDF files and the
coordinates of the marked fields to a file, which by default is named
``\hbox{posdata.txt}''.


\subsection{Architecture of the Form Filling UI}

Problematically, Ruby is the language in which the other components of this
system are implemented, but it lacks good support for PDF processing and user
interfaces. Currently I find that Node.js is better for the latter two, because
of Mozilla's pdf.js library and Nodegui (which, as of 2021, is still in
development but workable). To bridge the gap, the UI script actually runs in two
parts: A Ruby wrapper that manages the data files, and a Node-based UI. The two
communicate via Unix anonymous pipes, transmitting JSON objects between each
other.

The Nodegui dependencies are not included in the repository. To install them,
run \texttt{npm install} in the \texttt{taxcalc/ui} directory. The version
numbers in the \texttt{package.json} file should probably be updated first and
the \texttt{package-lock.json} file removed so it can be rebuilt.

If, in the future, Nodegui proves unworkable, it likely would not be difficult
either to turn the UI component into an Electron app or to switch to a different
framework entirely, retaining the Ruby wrapper as-is.


\subsection{Split Lines}

One feature that the marking script supports is ``split'' lines, where a line's
text is separated across multiple boxes (for example, one per letter or dividing
a social security number across dashes). The ``split'' checkbox on the UI
accommodates this.

Internally, a split line is represented by a flag in the Ruby class
Marking::Line. It cannot be an inherited class in order to enable switching
a line from split to non-split without creating a new object.




