\section{Informational Forms}



\subsection{Alimony}

Alimony received. See Pub.\ 504.

\begin{description}
\item[amount] The amount of alimony received.
\end{description}


\subsection{Asset}

A form for a capitalized asset.

\begin{description}
\item[date] The date the asset was put into service.
\item[amount] The dollar value of the asset when purchased.
\item[179?] Whether the asset is a section 179 deductible asset. Generally,
tangible personal property and computer software qualify. See Pub.~946 chapter
2.
\item[listed?] Whether the asset is a listed asset, see Pub.~946 chapter 5.
\item[dc\_category] The depreciation category for the asset in DC, see the FP-31
instructions, under Depreciation Guidelines.
\item[dc\_type] Choose from ``reference'', ``fixed'', or ``other'' to categorize
the asset for DC Form FP-31, lines 1--3.
\item[description] A description of the asset.
\end{description}


\subsection{Biographical}

Biographical information for an individual tax filer. Spouses should have a
separate Biographical form.

\begin{description}
\item[whose] The individual to whom this form pertains. May be ``mine'' or
``spouse''.
\item[first\_name] The first name and middle initial.
\item[last\_name] The last name.
\item[phone] Telephone number.
\item[ssn] Social Security number, with dashes.
\item[home\_address] The first line of the home address.
\item[apt\_no] The apartment number for the address.
\item[city\_zip] The city, state, and ZIP code.
\item[birthday] The person's birthday.
\item[blind?] Whether the person is blind.
\item[occupation] Your occupation
\end{description}

\subsection{Business Expense}

A deductible business expense. See IRS Pub.~535, chapter 11.

\begin{description}
\item[date] The date of the business expense.
\item[amount] The dollar amount of the expense.
\item[category] The category of expense. ``Meals'' and ``Utilities'' will
automatically be halved, the former per the IRS 50\% rule and the latter on the
assumption that 50\% of the expense was for non-business purposes. Other common
values are ``Supplies'', ``Travel'', and ``Membership''.
\item[description] A description of the expense.
\end{description}


\subsection{Charity Gift}

A charitable contribution.

\begin{description}
\item[amount] The dollar value of the contribution.
\item[cash?] Whether the contribution was in cash (as opposed to in-kind).
\item[name] The name of the charity.
\item[documented?] Whether the donation was documented. See Pub.\ 526. This is
checked if the contribution is over the \$250 threshold. Currently it is assumed
that any documentation meets the requirements of a contemporaneous written
acknowledgment (received before filing the return).
\end{description}
For charity gifts that are non-cash that total over \$500, Form 8283 must also
be filled, which requires additional information on the nature of the gift:
\begin{description}
\item[vin] The VIN of a donated vehicle.
\item[description] A description of the donated goods and their condition.
Clothing must be donated in good used condition.
\item[date] The date the contribution was made.
\item[date\_acq] The approximate date on which the property was acquired, or
``Various'' if the items were acquired on various dates at least 12 months ago.
\item[how\_acq] How the property was acquired, for example by purchase, gift,
inheritance, or exchange.
\item[basis] The basis or cost of the property.
\item[method] Method for determining fair market value. IRS examples are
appraisal, thrift shop value, catalog, or comparable sales.
\end{description}


\subsection{Dependent}

A dependent.

\begin{description}
\item[name] The dependent's name.
\item[ssn] The dependent's social security number.
\item[relationship] The dependent's relationship with the form provider.
\item[qualifying] Whether the dependent qualifies for a tax credit. Options are
``child'' for the child tax credit, ``other'' for the credit for other
dependents, and ``none'' where the dependent is not qualifying.
\item[birthday] The dependent's birthday.
\end{description}


\subsection{Dependent Care Benefit Use}

Information on the use of dependent care FSA benefits.

\begin{description}
\item[last\_year\_grace\_period\_use] The amount carried over from last year's
plan that was used this year.
\item[this\_year\_unused] The amount forfeited or carried over to next year.
\item[max\_contrib] The maximum contribution allowed for the FSA.
\end{description}

Any expenses that used this dependent care FSA should be given using a Dependent
Care Provider form, below.

\subsection{Dependent Care Provider}

Information on dependent care providers.

\begin{description}
\item[name] The name of the care provider.
\item[address] The address of the care provider.
\item[tin] The tax ID number (EIN or SSN) of the provider. May be "Tax-Exempt"
if appropriate.
\item[employee?] Whether the provider was a household employee.
\item[amount] The amount paid to this provider.
\item[dep\_ssn] The SSN of the child or dependent for whom this amount was paid.
\end{description}


\subsection{Estimated Tax}

Estimated tax payment made to the IRS.

\begin{description}
\item[amount] The amount of estimated tax paid.
\end{description}

\subsection{HSA Contribution}

Records of contributions to an HSA account.

\begin{description}
\item[ssn] The SSN of the beneficiary of the HSA account.
\item[contributions] Contributions made to the HSA, but not including employer
contributions or HSA/Archer MSA rollovers or HSA funding distributions.
\item[from] Either ``employer'' or ``self'' depending on who made the
contribution.
\end{description}

\subsection{HSA Excess Withdrawal}

Withdrawal of an excess HSA contribution. This should be entered in the tax year
for which the excess withdrawal applied, even if the withdrawal occurred in the
following year (Form 8889 automatically reaches into the previous year's records
to compute taxes on earnings and such).

\begin{description}
\item[date] The date that the excess withdrawal was actually performed.
\item[basis\_amount] The HSA contribution amount that was withdrawn.
\item[earnings\_amount] The amount of earnings on the withdrawn contribution.
\end{description}

\subsection{Home Office}

A portion of a home used as an office for a particular business.

\begin{description}
\item[type] The type of business associated with this home office. Currently the
only supported type is ``partnership''.
\item[ein] The EIN of the relevant business.
\item[method] Either ``simplified'' or ``actual''.
\item[sqft] Square footage of the home office area.
\item[daycare?] Whether the relevant business was a daycare.
\item[property] The name of the Real Estate form with which this home office is
associated.
\end{description}

\subsection{State Estimated Tax}

Estimated tax payment made to a state.

\begin{description}
\item[amount] The amount of estimated tax paid.
\item[state] The two-letter code for the state.
\end{description}




\subsection{State Tax}

A record of state taxes paid in the relevant tax year (usually for the previous
year). This is the amount actually paid to the state tax office, (i.e., the
amount due), not the total tax including what was already paid via withholdings.

\begin{description}
\item[amount] The amount of tax paid.
\item[name] A description (not used for computation).
\end{description}

\subsection{Traditional IRA Contribution}

A contribution to a traditional IRA.

\begin{description}
\item[amount] The amount contributed.
\item[tax\_year] The tax year of the contribution.
\item[date] The date of the contribution.
\end{description}


\subsection{Partner}

A single partner within a partnership.

\begin{description}
\item[name] The partner's name.
\item[liability] Either ``general'' or ``limited''.
\item[nationality] Either ``domestic'' or ``foreign''.
\item[type] The entity type of the partner. Currently only ``Individual'' is
supported fully. Other possibilities include ``Estate''. (This should also, in
the future, include corporations, partnerships, trusts, nonprofits, and foreign
governments, among others.)
\item[share] This partner's share of profits and losses, as a decimal fraction
such that all the partners' shares add to 1. Currently it is assumed that shares
of profits and shares of losses are equal.
\item[capital] This partner's percentage contribution of capital to the
partnership, as a decimal fraction such that all the partners' shares add to 1.
\item[ssn] The SSN of the partner.
\item[ein] The EIN of the partnership.
\item[country] The country of citizenship of the partner, relevant to Form 1065
Schedule B-1, part II, column iii.
\item[address] The first line of the partner's address.
\item[address2] The second line of the partner's address.
\item[active?] Whether the partner was active in the partnership's business.
\end{description}




\subsection{Partnership}

Biographical information for a partnership.

\begin{description}
\item[name] The name of the partnership.
\item[address] The first line of the address.
\item[address2] The second line of the address.
\item[business] The line of business of the partnership, Form 1065 line A.
\item[product] The product or service of the partnership, Form 1065 line B.
\item[code] The business code, Form 1065 line C.
\item[ein] The employer identification number of the partnership.
\item[start] The start date of the partnership.
\item[accounting] ``Cash'', ``Accrual'', or other accounting method.
\end{description}

\subsection{Real Estate}

A description of real property, used in a variety of contexts and to link other
forms that deal with a particular property.

\begin{description}
\item[property] A name for the property that will uniquely identify the property
across tax forms.
\item[basis] The basis price for the property, i.e., the purchase value. See
Pub.\ 551.
\item[sqft] The square footage of the property.
\item[purchase\_date] The date on which the property was purchased.
\end{description}




